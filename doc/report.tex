\documentclass[11pt]{article}
\usepackage[UTF8]{ctex}
\usepackage[T1]{fontenc}
\usepackage[colorlinks, linkcolor=black]{hyperref}
\usepackage{indentfirst}
\usepackage{babel}
\usepackage{color}
\usepackage[a4paper,left=3cm,right=3cm,top=2.5cm,bottom=2.5cm]{geometry}
\usepackage[toc,page]{appendix}
\usepackage{minted}
\usepackage[htt]{hyphenat}

\title{\textbf{数据库系统概论项目报告}}
\author{
    潘子睿\\
    \texttt{2020010960}
    \and
    余任杰\\
    \texttt{2020010966}
}

\begin{document}
\maketitle
\tableofcontents
\setlength{\parindent}{0pt}
\clearpage

\section{项目架构}
\texttt{SimDB}是一个简单的关系型数据库系统,其能够支持一些基本的\texttt{SQL}语句,完成对数据库的插入、删除、更新、查找等操作。
项目的最底层是一个页式文件管理系统,在其上构建了记录管理和索引管理两个模块,分别用来维护数据库中的一条条记录以及在某些记录上建立的索引。
系统管理模块通过调用记录管理和索引管理两个模块的接口,实现了具体操作数据库的各项功能。
最后,通过\texttt{Antlr4}解析\texttt{SQL}指令,并将其交给系统管理模块执行。\\\\
本项目源代码目前位于\texttt{https://github.com/pzrain/SimDB},并将在本课程结课后开源。
\subsection{支持功能}
\subsubsection{数据结构}
数据库当前支持的数据结构包括:
\begin{itemize}
    \item 整型(\texttt{INT})
    \item 浮点型(\texttt{FLOAT})
    \item 字符串型(\texttt{VARCHAR})
\end{itemize}
其中\texttt{VARCHAR}仅支持定长字符串。
\subsubsection{SQL语句}
\subsubsection{其他}
\subsection{代码测试}

\section{编译与运行}
本项目基于\texttt{CMake}进行自动构建。需要\texttt{Antlr4}依赖,以及编译器支持\texttt{C++17}特性。
使用时,将\texttt{CMakeLists.txt}中的\texttt{ANTLR4\_RUNTIME\_DIRECTORY}的值设置为\texttt{Antlr4}运行时库\texttt{antlr4-runtime.h}所在的目录,并将\texttt{ANTLR4\_RUNTIME\_SHARED\_LIBRARY}的值设置为Antlr4运行时动态链接库\texttt{libantlr4-runtime.so}。\\
成功配置好\texttt{Antlr4}后,在项目根目录下执行以下命令:
\begin{minted}{bash}
./run.sh -c
\end{minted}
即可自动进行编译并运行,生成的可执行文件\texttt{SimDB}会被放置在\texttt{./bin}目录下。

\section{系统功能}
\subsection{页式文件系统}
本部分代码位于\texttt{./src/filesystem},需要注意的是,本部分直接使用了课程实验文档附录提供的参考实现代码。\\\\
数据库是被设计用来存储大量数据的系统,数据库中一个文件的大小甚至可能超过计算机的内存。
因此,需要一个页式文件管理系统来管理数据库的各个文件,以及一个缓存机制,将操作的多个页面缓存在内存中,只在需要时进行替换和写回,以提高读写的效率。
参考实现中使用的替换算法为最近最少使用算法(\texttt{LRU})。
\subsection{记录管理}
\subsection{索引管理}
\subsection{系统管理}
\subsection{SQL解析}

\section{参考资料}
以下列出了本项目完成过程中所参考的部分资料,包括课程的实验文档、往届的实现,以及相关网站等。
\begin{itemize}
    \item \href{https://github.com/miskcoo/TrivialDB}{\textcolor{black}{往届实现的数据库项目}}
    \item \href{https://web.stanford.edu/class/cs346/2015/redbase.html}{\textcolor{black}{CS346 Redbase project}}
    \item \href{https://en.wikipedia.org/wiki/B%2B_tree}{\textcolor{black}{$B^+$树的定义与实现}}
    \item \href{https://www.antlr.org/download.html}{\textcolor{black}{使用antlr4解析SQL}}
    \item \href{https://thu-db.github.io/dbs-tutorial/}{\textcolor{black}{课程实验文档}}
\end{itemize}

\end{document}