\documentclass[11pt]{article}
\usepackage[UTF8]{ctex}
\usepackage[T1]{fontenc}
\usepackage[colorlinks, linkcolor=black]{hyperref}
\usepackage{indentfirst}
\usepackage{babel}
\usepackage{color}
\usepackage[a4paper,left=3cm,right=3cm,top=2.5cm,bottom=2.5cm]{geometry}
\usepackage[toc,page]{appendix}
\usepackage{minted}
\usepackage[htt]{hyphenat}
\usepackage{tikz}
\usepackage{bytefield}

\title{\textbf{数据库系统概论项目报告}}
\author{
    潘子睿\\
    \texttt{2020010960}
    \and
    余任杰\\
    \texttt{2020010966}
}

\begin{document}
\maketitle
\tableofcontents
\setlength{\parindent}{0pt}
\clearpage

\section{项目架构}
\texttt{SimDB}是一个简单的关系型数据库系统,其能够支持一些基本的\texttt{SQL}语句,完成对数据库的插入、删除、更新、查找等操作。
项目的最底层是一个\textbf{页式文件管理系统},在其上构建了\textbf{记录管理}和\textbf{索引管理}两个模块,分别用来维护数据库中的一条条记录以及在某些记录上建立的索引。
\textbf{系统管理}模块通过调用记录管理和索引管理两个模块的接口,实现了具体操作数据库的各项功能。
最后,在\textbf{查询解析}模块中通过\texttt{Antlr4}解析\texttt{SQL}指令,并将其交给系统管理模块执行。\\\\
本项目源代码目前位于\texttt{https://github.com/pzrain/SimDB},并可能会在本课程结课后开源。
\subsection{支持功能}
\subsubsection{数据结构}
数据库当前支持的数据结构包括:
\begin{itemize}
    \item 整型(\texttt{INT})
    \item 浮点型(\texttt{FLOAT})
    \item 字符串型(\texttt{VARCHAR})
\end{itemize}
其中\texttt{VARCHAR}仅支持定长字符串。
\subsubsection{SQL语句}
\subsubsection{其他}
\subsection{代码测试}

\section{编译与运行}
本项目基于\texttt{CMake}进行自动构建。需要\texttt{Antlr4}依赖,以及编译器支持\texttt{C++17}特性。
使用时,将\texttt{CMakeLists.txt}中的\texttt{ANTLR4\_RUNTIME\_DIRECTORY}的值设置为\texttt{Antlr4}运行时库\texttt{antlr4-runtime.h}所在的目录,并将\texttt{ANTLR4\_RUNTIME\_SHARED\_LIBRARY}的值设置为Antlr4运行时动态链接库\texttt{libantlr4-runtime.so}所在位置。\\
成功配置好\texttt{Antlr4}后,在项目根目录下执行以下命令:
\begin{minted}{bash}
./run.sh -c
\end{minted}
即可自动进行编译并运行,生成的可执行文件\texttt{SimDB}位于在\texttt{./bin}目录下。

\section{系统功能}
\subsection{页式文件系统}
本部分代码位于\texttt{./src/filesystem},需要注意的是,本部分直接使用了课程实验文档附录中提供的参考实现代码(对其中部分接口做了略微修改)。\\\\
数据库是被设计用来存储大量数据的系统,数据库中一个文件的大小甚至可能超过计算机的内存。
因此,需要一个\textbf{页式文件管理系统}来管理数据库的各个文件,以及一个\textbf{缓存}机制,将操作的多个页面缓存在内存中,只在需要时进行替换和写回,以提高读写的效率。
参考实现中使用的替换算法为\textbf{最近最少使用算法}(\texttt{LRU})。

\subsection{记录管理}
记录管理模块是整个数据库系统中相对比较底层的模块之一。其负责管理存入数据库的一条条记录,
具体而言,记录管理模块需要支持的功能包括:将某一记录存放在某一文件的特定位置处,并维护好该记录的位置信息、是否为空的标记等;
根据指定的位置,从存放数据的文件中取出指定的记录;将某一指定的记录从文件中删除;
修改文件中某一指定位置的记录,实际上就是先读取记录,修改后再写回到原来的位置。\\

同时,本模块还负责维护一张表的结构,具体而言,需要支持的功能包括:增加一张表;删除一张表;给原有的表增加/删除一个表项;修改原有的表中的一个表项等。
为了实现以上的所有功能,记录管理模块会调用页式文件系统中定义的各个接口,从而完成实际的文件\texttt{I/O}操作。\\

在本模块的具体实现中,一个数据库以文件夹的形式存储,该数据库下的一张表以单个文件的形式存放在对应文件夹下。 
在处理某一文件中的多个内存页时,本模块将其分为两类,分别为表头页和记录页。
前者用来存放一张表以及其对应文件的\textbf{元数据},包括表的列数,每列的具体要求,以及该文件的总页数,第一个有空闲位置的页等信息。
后者则用来存放对应表下的具体记录。

\subsubsection{记录结构}
一条记录包含多个表项的具体内容。记录实际上存在两种结构,分别为\textbf{序列化}和\textbf{反序列化}后的结果,实际存储在文件中的记录是序列化后的结果。
\paragraph{序列化} 序列化后的结果为字节的序列,也即将记录中的各个表项拼接在一起,组成一个\texttt{char}数组。
该字符数组的前两位字节被用来判断记录中各个表项是否为空值(\texttt{Null})。
\paragraph{反序列化} 反序列化即为将字节的序列整理成为更易操作的格式,实际过程中对记录的修改都是基于反序列化的形式。
反序列化会将字节数组的各个表项提取出来,构造成一个链表。
\paragraph{记录位置} 每条记录的位置被维护成一个\textbf{二元组\texttt{(pageId, slotId)}},表示该记录被存放在第\texttt{pageId}页上的第\texttt{slotId}槽中。

\subsubsection{表头页结构}
每个文件对应的第一个内存页被处理成\textbf{表头页}。表头页中存储的各个字段如下\\
% uint8_t valid;
% uint8_t colNum;
% int8_t entryHead;
% int16_t firstNotFullPage;
% uint16_t recordLen, totalPageNumber; // recordLen: length of one record
% uint32_t recordSize; // number of records/slots on one page
% uint32_t recordNum; // total number of records;
% TableEntry entrys[TAB_MAX_COL_NUM];
% char tableName[TAB_MAX_NAME_LEN];
\begin{center}
    \begin{tabular}{|c|c|c|}
        \hline
            \textbf{名称}                  & \textbf{占用字节数} & \textbf{描述}\\
        \hline 
            valid                          & 1                  & 该页是否已经初始化\\
        \hline 
            colNum                         & 1                  & 表中的项数\\
        \hline 
            entryHead                      & 1                  & 第一项在entrys中对应的下标\\
        \hline
            firstNotFullPage               & 2                  & 第一个非满页的页码\\
        \hline
            recordLen                      & 2                  & 表中定长记录的长度\\
        \hline
            totalPageNumber                & 2                  & 当前总页数\\
        \hline
            recordSize                     & 4                  & 一页上所能存放的记录数\\
        \hline
            recordNum                      & 4                  & 总记录数\\
        \hline
            entrys[TAB\_MAX\_COL\_NUM]     & -                  & 各表项的具体描述\\
        \hline
            tableName[TAB\_MAX\_NAME\_LEN] & -                  & 各表项的名称\\
        \hline
    \end{tabular}  
\end{center}
其中,每个表项使用一个类\texttt{TableEntry}来描述,表中的\texttt{entrys}即为\texttt{TableEntry}的数组。
\texttt{TableEntry}类的具体实现,参见附录中的各模块接口详细说明。
    
\subsubsection{记录页结构}
文件对应的各个内存页中,除了第一个为表头页外,其余均为\textbf{记录页}。记录页的结构组织如下:\\
\begin{bytefield}[bitwidth=.125\linewidth, bitheight=7mm]{8}
    \bitbox[]{4}[]{\noindent31\hfill16\quad} & \bitbox[]{4}[]{\quad15\hfill0}\\
    \bitboxes{4}[bgcolor=yellow!30]{{nextFreePage} {firstEmptySlot}} \\
    \bitboxes{4}[bgcolor=blue!30]{{totalSlot} {maximumSlot}} \\
    \bitbox{1}[bgcolor=gray!30]{slotHead}\bitbox{7}[bgcolor=gray!30]{record}\\
    \bitbox{8}[bgcolor=gray!30]{...}\\
    \bitbox{1}[bgcolor=gray!30]{slotHead}\bitbox{7}[bgcolor=gray!30]{record}
\end{bytefield}\\
记录页中的\texttt{nextFreePage},以及表头页中的\texttt{firstNotFullPage},将所有已分配的空闲页串成了一张链表。这样,在插入一条记录时,可以迅速找到有空闲位置的页;删除记录时,如果该页上的记录已经被删完了,就将其添加到空闲页链表的尾部。
\texttt{totalSlot}记录的为当前页面内记录的总数,如果其达到了\texttt{maximumSlot},说明页面已被填满,需要将其从空闲页链表中去除。
对于非空闲页,该域中的内容可以是任意值。\\

如上图所示,记录页中的前8个字节用于记录和空闲页管理以及页内槽数相关的信息,剩下的空间中会\textbf{紧密}排列着各条记录。
由于记录是定长的,因此一旦记录的长度确定,就可以计算出页面上最多能存放的记录总数,以及每条记录存放位置的偏移。
与空闲页的管理类似,空闲槽也被组织成链表,对任一个空闲槽,其对应的下一个空闲槽的编号被记录在\texttt{slotHead}中。链表尾部空闲槽该域的值为\texttt{-1}。
而对于非空闲(已经写入了记录的)的槽,其\texttt{slotHead}域会被记录为\texttt{SLOT\_DIRTY}以进行区分。
\subsection{索引管理}
\subsection{系统管理}
\subsection{查询解析}

\section{项目分工}
\begin{itemize}
    \item 记录管理、索引管理:潘子睿
    \item 系统管理、查询解析:余任杰
\end{itemize}

\section{参考资料}
以下列出了本项目完成过程中所参考的部分资料,包括课程的实验文档、往届的实现,以及相关网站等。
\begin{itemize}
    \item \href{https://github.com/miskcoo/TrivialDB}{\textcolor{black}{往届实现的数据库项目}}
    \item \href{https://web.stanford.edu/class/cs346/2015/redbase.html}{\textcolor{black}{CS346 Redbase project}}
    \item \href{https://en.wikipedia.org/wiki/B%2B_tree}{\textcolor{black}{$B^+$树的定义与实现}}
    \item \href{https://www.antlr.org/download.html}{\textcolor{black}{使用\texttt{Antlr4}完成\texttt{SQL}的语法解析}}
    \item \href{https://thu-db.github.io/dbs-tutorial/}{\textcolor{black}{课程实验文档}}
\end{itemize}

\end{document}